\chapter{Language Modeling and Pretraining Objectives}
\label{chap:lm_pretraining}
% PROMPT: Differentiate cross-entropy loss from perplexity

\noindent
\textbf{Language modeling} in NLP involves predicting or representing textual data in a way that captures statistical regularities, semantics, and contextual relationships among tokens. In modern LLMs, language modeling objectives underpin pretraining, enabling models to learn general-purpose linguistic representations from large-scale corpora. This chapter discusses the key objectives used in language modeling and how they shape both the training process and the learned representations.

\section{Statistical Language Modeling}
\label{sec:stat_lm}
% PROMPT: Differentiate cross-entropy loss from perplexity

\subsection{Maximum Likelihood Estimation (MLE) for Next-Token Prediction}
\noindent
A traditional approach to language modeling is to maximize the likelihood of the observed sequence of tokens under the model’s parameters. Given a training corpus of token sequences, we denote each sequence by \(\mathbf{w} = (w_1, w_2, \ldots, w_T)\). Under the \textbf{chain rule} of probability, the joint probability of the sequence is:
\[
P(\mathbf{w}) = \prod_{t=1}^{T} P(w_t \mid w_1, w_2, \ldots, w_{t-1}).
\]
To train the model, we often minimize the negative log-likelihood of the data:
\[
-\log P(\mathbf{w}) = -\sum_{t=1}^{T} \log P(w_t \mid w_1, \ldots, w_{t-1}),
\]
which is equivalent to maximizing \(P(\mathbf{w})\) under maximum likelihood estimation (MLE). 

\subsection{Cross-Entropy Loss and Perplexity}
\noindent
In practice, we implement this via the \textbf{cross-entropy loss}, commonly denoted as:
\[
\mathcal{L}_\text{CE} = -\sum_{t=1}^{T} \log \Bigl( P_{\theta}(w_t \mid w_1, \ldots, w_{t-1}) \Bigr),
\]
where \(P_{\theta}\) is the model distribution parameterized by \(\theta\). Cross-entropy compares the model’s predicted distribution over tokens with the true one-hot distribution, penalizing the model when it assigns low probability to the correct token.

\begin{itemize}
    \item \textbf{Perplexity.}
    A common metric for language model performance, \emph{perplexity} (PPL) is defined as:
    \[
    \text{PPL} = \exp\!\Bigl(\frac{1}{T}\sum_{t=1}^T -\log P_{\theta}(w_t \mid w_{<t})\Bigr).
    \]
    Conceptually, perplexity measures the average branching factor of the model’s predictive distribution. A lower perplexity indicates that the model is less “surprised” by the data. 
    \item \textbf{Cross-Entropy vs. Perplexity.}
    While cross-entropy \(\mathcal{L}_\text{CE}\) is often used directly as the training objective, perplexity provides an intuitive, exponential-scale view of the model’s uncertainty on test data.
\end{itemize}


\section{Masked Language Modeling (MLM)}
\label{sec:mlm}
% PROMPT: Outline how partial masking affects training stability

\subsection{BERT-Style Masked Token Prediction}
\noindent
\textbf{Masked Language Modeling (MLM)} is a pretraining objective popularized by BERT. In MLM, a certain percentage of tokens (e.g., 15\%) in a sequence are replaced by a special \texttt{[MASK]} token or, less frequently, by random tokens. The model is then tasked with predicting the original tokens in these masked positions.

\[
\mathbf{w}_{\text{masked}} = (w_1, \ldots, \texttt{[MASK]}, w_{j}, \ldots),
\]
where \(w_i\) is replaced by \texttt{[MASK]} at selected positions \(i\). 

\begin{itemize}
    \item \textbf{Bidirectional Context.}
    Since the model can attend to tokens on both the left and right of a masked position, MLM captures bidirectional context more effectively than a strictly left-to-right approach.
    \item \textbf{Robustness.}
    By training to predict randomly masked tokens, the model learns more robust token representations, reducing overfitting to specific sequential patterns.
\end{itemize}

\subsection{The Math Behind Partial Conditioning}
\noindent
For each masked position \(i\), the model’s goal is to predict \(w_i\) given the remaining (unmasked) tokens:
\[
P(w_i \mid w_1, \ldots, \texttt{[MASK]}, \ldots, w_{j}, \ldots),
\]
where all but the masked tokens are visible. The loss is computed over just the masked positions, typically via a cross-entropy objective:
\[
\mathcal{L}_\text{MLM} = - \sum_{i \in M} \log P_{\theta}(w_i \mid \mathbf{w}_{\text{masked}}),
\]
where \(M\) is the set of masked indices. This \emph{partial conditioning} forces the network to infer the masked token using both left and right context, thus learning more powerful bidirectional representations.

\begin{itemize}
    \item \textbf{Training Stability.}
    Partial masking smooths the training signal by updating parameters based on a more varied set of contexts. The ability to predict one token given a surrounding window of unmasked tokens often leads to faster convergence and more robust representations.
\end{itemize}


\section{Causal Language Modeling (CLM)}
\label{sec:clm}
% PROMPT: Emphasize left-to-right modeling for generation tasks

\subsection{GPT-Style Left-to-Right Prediction}
\noindent
\textbf{Causal Language Modeling (CLM)} is an \emph{auto-regressive} objective in which the model predicts each token \(w_t\) based solely on the preceding tokens \(\{w_1, w_2, \ldots, w_{t-1}\}\). This left-to-right formulation is especially relevant for generative tasks, where the model iteratively generates tokens one after another.

\[
P(w_t \mid w_1, \ldots, w_{t-1}).
\]
\begin{itemize}
    \item \textbf{GPT Family.}
    Models like GPT, GPT-2, and GPT-3 use causal language modeling to generate text in a left-to-right fashion. Their impressive generative capabilities stem partly from extensive pretraining on large corpora with this objective.
\end{itemize}

\subsection{Training Objective and Computational Considerations}
\noindent
During training, the model maximizes the log-likelihood of each token given all previous tokens:
\[
\mathcal{L}_\text{CLM} = - \sum_{t=1}^{T} \log P_{\theta}(w_t \mid w_{<t}).
\]
\begin{itemize}
    \item \textbf{Efficiency.}
    Auto-regressive training can be parallelized over tokens within a sequence by shifting the inputs and targets (sometimes called the “teacher forcing” approach in RNNs). However, at inference time, tokens must be generated sequentially.
    \item \textbf{Long-Range Context.}
    Because each token attends to all previous tokens, the model can theoretically capture long-range dependencies. In practice, attention-based architectures allow for effective parallelization despite the left-to-right constraint.
\end{itemize}


\section{Next Sentence Prediction, Permutation LM, and Other Objectives}
\label{sec:other_obj}
% PROMPT: Compare objectives like next sentence prediction and permutation LM

\subsection{How Objectives Influence Model’s Learned Representations}
\noindent
Beyond MLM and CLM, a variety of pretraining objectives have been explored to capture different facets of language:

\begin{itemize}
    \item \textbf{Next Sentence Prediction (NSP).}
    Used in the original BERT, NSP asks the model to predict whether one segment of text logically follows another. This objective encourages the model to learn inter-sentence relationships, aiding tasks like question answering or reading comprehension. However, subsequent research has shown that NSP might not be as crucial as once thought; some models (e.g., RoBERTa) omit NSP altogether.

    \item \textbf{Permutation Language Modeling (XLNet).}
    XLNet generalizes auto-regressive modeling by permuting the factorization order of the tokens. This approach captures bidirectional context while retaining an auto-regressive training scheme, offering a blend of the benefits of MLM and CLM.

    \item \textbf{Other Task-Specific Objectives.}
    Further specialized objectives—like \emph{denoising autoencoders} in T5 (with random spans masked) or \emph{sentence-order prediction}—target particular downstream tasks or linguistic properties.

\end{itemize}

\noindent
Each objective shapes what the model learns about language. Models that rely on \emph{masked} or \emph{bidirectional} contexts often excel at understanding text (e.g., classification, question answering), whereas \emph{causal} models are more adept at generative tasks (e.g., text completion, story generation). However, many LLMs today blend or fine-tune multiple objectives for maximum flexibility.

\noindent
In summary, \textbf{pretraining objectives} provide the foundation for large-scale language model training, guiding how models acquire semantic understanding, context representation, and generative capabilities. By choosing or combining these objectives carefully, practitioners can tailor LLMs to excel across a wide spectrum of NLP tasks. 
