\section{Organization of the Book}
\label{sec:book_organization}
% PROMPT: Provide a roadmap of the book’s structure

\noindent
The purpose of this book is to provide a comprehensive overview of Large Language Models (LLMs) and the mathematics that underpins them, from foundational concepts to advanced techniques. To help you navigate the material, the book is divided into five main parts, each focusing on a key aspect of LLMs:

\begin{itemize}
    \item \textbf{Part I: Foundations.} 
    This section covers the essential mathematical background—including linear algebra, calculus, probability, and basic machine learning concepts—necessary to understand the more complex ideas presented later in the book.

    \item \textbf{Part II: Transformer Architecture and Attention.} 
    Here, we delve into the core building blocks of modern LLMs. We unpack the attention mechanism, explore how it is scaled up through multi-head attention, and examine the complete Transformer encoder-decoder structure that revolutionized NLP.

    \item \textbf{Part III: Training Large Language Models.}
    This section focuses on the training objectives for language modeling, large-scale optimization techniques, and practical considerations for scaling up to billions of parameters. We also discuss hyperparameter tuning strategies and the empirical scaling laws that guide model growth.

    \item \textbf{Part IV: Advanced Topics and Future Directions.}
    Beyond the essentials, we look at fine-tuning methods, prompt engineering, interpretability, and bias concerns. We also explore how LLMs can be distilled or compressed for efficient deployment, and discuss the latest frontiers like multimodal models and reinforcement learning integrations.

    \item \textbf{Part V: Conclusion and Resources.}
    The book concludes with a summary of the key mathematical concepts and practical takeaways, along with pointers to additional resources such as research papers, code libraries, and online courses. 

\end{itemize}

\noindent
Throughout each part, you will find both theoretical explanations and practical implementations. Where possible, we provide step-by-step derivations of key formulas, code snippets, and hands-on exercises to reinforce learning. By the end of the book, our aim is for you to have both a solid mathematical grounding in LLMs and the practical know-how to experiment with—and even innovate upon—this rapidly evolving class of models.
