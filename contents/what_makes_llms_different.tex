\section{What Makes LLMs Different?}\index{LLM!differences}\index{language models!large vs small}
\label{sec:llms_difference}
% PROMPT: Compare small language models to large language models in more detail

\noindent
While traditional language models\index{language models!traditional}—be they n-gram\index{n-gram models} or smaller neural architectures\index{neural architectures!small}—have contributed substantially to various NLP tasks\index{NLP tasks}, \textbf{Large Language Models (LLMs)}\index{LLM|see {Large Language Model}}\index{Large Language Model} stand apart in terms of scale\index{scale}, performance\index{performance}, and versatility\index{versatility}. This section explores some of the key attributes that differentiate LLMs from their predecessors\index{model predecessors}.

\begin{itemize}
    \item \textbf{Scale and Parameter Count.}\index{scale!parameters}\index{parameter count}
    One of the most defining characteristics of an LLM is its sheer size\index{model size}. These models can contain billions\index{parameters!billions} (or even trillions\index{parameters!trillions}) of parameters, far surpassing the capacity of earlier architectures\index{model capacity}. Such expansive parameter spaces\index{parameter space} allow LLMs to store rich linguistic\index{linguistic information} and factual information\index{factual information}, leading to improved performance on diverse tasks\index{task diversity}, from text classification\index{text classification} to creative text generation\index{text generation!creative}. However, training\index{training!computational requirements} and serving\index{model serving} these enormous models come with steep computational\index{computational requirements} and memory requirements\index{memory requirements}, motivating significant research into distributed systems\index{distributed systems} and model parallelism\index{model parallelism}.

    \item \textbf{Data Requirements and Training Pipelines.}\index{data requirements}\index{training pipelines}
    LLMs are typically pre-trained\index{pre-training} on massive text corpora\index{text corpora} scraped from the web\index{web scraping}, encompassing a wide variety of styles\index{text styles}, domains\index{domains}, and languages\index{languages}. This \emph{pre-training phase}\index{pre-training phase} enables the model to learn general-purpose representations\index{representations!general-purpose} of language before being adapted to specific tasks. The construction of such large-scale datasets\index{datasets!large-scale} often involves:
    \begin{itemize}
        \item Web crawling\index{web crawling} and cleaning\index{data cleaning} to remove noise\index{noise removal} and inappropriate content\index{content filtering}.
        \item Tokenization\index{tokenization} or text normalization\index{text normalization} steps to convert raw text into manageable tokens\index{tokens}.
        \item Automated filtering\index{automated filtering} to reduce duplication\index{duplication reduction} and protect privacy\index{privacy protection} where possible.
    \end{itemize}
    This pre-training pipeline\index{pre-training pipeline}, coupled with advanced optimization methods\index{optimization methods}, allows LLMs to \emph{self-supervise}\index{self-supervision} on unlabeled text\index{unlabeled text}, capturing vast contextual\index{contextual knowledge} and semantic knowledge\index{semantic knowledge}.

    \item \textbf{Impact on Natural Language Tasks and Applications.}\index{natural language tasks}\index{applications}
    Once pre-trained, LLMs can be fine-tuned\index{fine-tuning} or prompted\index{prompting} to perform exceptionally well on a broad array of downstream tasks\index{downstream tasks}:
    \begin{itemize}[noitemsep]
        \item \textit{Reading Comprehension \& Question Answering:}\index{reading comprehension}\index{question answering} LLMs can interpret context passages\index{context interpretation} and provide answers with a depth that surpasses smaller models.
        \item \textit{Text Generation \& Summarization:}\index{text generation}\index{summarization} By leveraging expansive learned knowledge\index{learned knowledge}, LLMs produce cohesive summaries\index{summaries} and human-like text\index{human-like text} in various domains.
        \item \textit{Zero-Shot \& Few-Shot Learning:}\index{zero-shot learning}\index{few-shot learning} Owing to their large pre-training corpus, LLMs can tackle new tasks with minimal or even no explicit examples, demonstrating emergent generalization skills\index{generalization skills}.
        \item \textit{Creative Tasks:}\index{creative tasks} LLMs' generative prowess\index{generative capabilities} allows them to assist in writing prose\index{prose writing}, poetry\index{poetry}, or even computer code\index{code generation}, showcasing impressive degrees of fluency\index{fluency} and coherence\index{coherence}.
    \end{itemize}
    This versatility increasingly positions LLMs as a \emph{universal backbone}\index{universal backbone} for NLP applications\index{NLP applications}, transforming how researchers and practitioners approach natural language tasks.

    \item \textbf{Challenges and Considerations.}\index{challenges}\index{considerations}
    With great power comes great responsibility. LLMs raise pressing concerns around:
    \begin{enumerate}
        \item \textbf{Ethical and Bias Issues:}\index{ethical issues}\index{bias issues} LLMs can inadvertently learn and propagate societal biases\index{societal biases} present in their training data, creating potential harm\index{potential harm} in real-world applications\index{real-world applications}.
        \item \textbf{Computational Costs:}\index{computational costs} Training and deploying LLMs require considerable computational resources\index{computational resources}, making them less accessible\index{accessibility} to smaller research groups\index{research groups} and increasing the environmental footprint\index{environmental footprint}.
        \item \textbf{Interpretability and Control:}\index{interpretability}\index{control} As models grow more complex\index{model complexity}, understanding their decision-making processes\index{decision-making} becomes challenging, raising questions about transparency\index{transparency} and reliability\index{reliability}.
    \end{enumerate}
    These challenges underscore the necessity for ongoing research in techniques like parameter-efficient fine-tuning\index{parameter-efficient fine-tuning}, model compression\index{model compression}, bias detection\index{bias detection}, and interpretability frameworks\index{interpretability frameworks}.

\end{itemize}

\noindent
In sum, LLMs differentiate themselves through their capacity to learn universal linguistic representations from massive data, adapt quickly to new tasks, and generate highly coherent text. Their ever-growing size, however, entails significant engineering, ethical, and societal considerations, making it crucial for future work to balance innovation with responsible deployment. 
